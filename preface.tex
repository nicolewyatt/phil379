\chapter{Preface}

A formal logic consists of a symbolic language together with a semantics, which captures the possible meanings or truth-conditions of the sentences of the language, and a deductive system, which aims to capture which inferences are correct. In this course we study the scope and limits of formal logic by examining the relationship between these three parts of a logic. The major results to be presented include soundness (``the deductive system captures only truths''), completeness (``the deductive system captures all the truths''), undecidability (``there is no mechanical procedure for establishing whether or not an argument is valid''), and the L\"owenheim-Skolem theorems (which concern some of the limits on the expressive power of first-order logic). Along the way we will study some set theory, Turing machines, the limits of computation, and some of the philosophical motivations for the development of first-order logic in the early twentieth century. The course is fast-paced and students will need to supplement the lectures with independent study. 

 
\begin{description}
\item[Week 1] (Jan 13). Introduction. Logic and mechanical procedures. \textbf{No class Jan 15}

\item[Week 2] (Jan 20, 22). Sets, Relations, Functions. Enumerability. \emph{Assignment 1 due Jan 22}

\item[Week 3] (Jan 27, 29). Syntax and Semantics of FOL. \emph{Assignment 2 due Jan 29}

\item[Week 4] (Feb 3, 5). Syntax and Semantics of FOL continued. \emph{Assignment 3 due Feb 5}

\item[Week 5] (Feb 10, 12). Sequent Calculus and Proofs in FOL. \emph{Assignment 4 due Feb 12}

\item[Reading week] No classes Feb 16-20.

\item[Week 6] (Feb 24). Introduction to soundness and completeness. \emph{Midterm exam Feb 26th}

\item[Week 7] (Mar 3, 5). Completeness Proofs

\item[Week 8] (Mar 10, 12). Compactness and L\"owenheim-Skolem Theorems \emph{Assignment 5 due Mar 12}

\item[Week 9] (Mar 17, 19). Computability and Turing Machines \emph{Assignment 6 due Mar 19}

\item[Week 10] (Mar 24, 26). Turing machines continued. \emph{Assignment 7 due Mar 26}

\item[Week 11] (Mar 31). The Church-Turing Thesis. \textbf{No class April 2} \emph{Assignment 8 due April 2}

\item[Week 12] (Apr 7, 9). Undecidability

\item[Week 13] (Apr 14). Undecidability continued. \textbf{Final exam will be scheduled by the registrar}
\end{description}

